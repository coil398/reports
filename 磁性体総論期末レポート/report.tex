\documentclass{jsarticle}
\usepackage[dvipdfmx]{graphicx}
\begin{document}

\title{磁性体総論期末レポート}
\author{1518511 川瀬 拓実}
\maketitle

\newpage

\section{論文について}
今回読んだ論文は以下のものである。
The Structural Consequences of Charge Disproportionation in Mixed-Valence Iron Oxides. 1. The Crystal Structure of $\rm{Sr_{2}LaFe_{3}O_{8.94}}$ at Room Temperature and 50K
本論文ではいくつかのperovskite構造を持つ物質について議論されているが、本題の$\rm{Sr_{2}LaFe_{3}O_{8.94}}$の議論のために補助的に利用されているだけなので、あくまで$\rm{Sr_{2}LaFe_{3}O_{8.94}}$について要約する。

\section{要約}
メスバウアー効果の測定によると$\rm{Sr_{2}LaFe_{3}O_{8.94}}$において鉄原子の電子に低温で反強磁性的な電荷不均化状態が起こることがわかっている。更に、その相転移温度は酸素の量に依存しており、実際に酸素を量を変えることで相転移温度が変わることが実験的に明らかにされている。
電荷不均化状態での鉄原子は以下のように相転移する。
\begin{eqnarray}
3Fe^{3.66+} \leftrightarrow 2Fe^{3+} + Fe^{5+} \nonumber
\end{eqnarray}
粉末X線回折と中性子粉末回折による実験を室温と50Kで行った。

室温においては、立方対称からの歪みによって高角度の線に広がりが見える。また、中性子のデータから構造は$\overline{R}3c$もしくは六方晶で記述される。
ランタンとストロンチウムは完全に無秩序に並んでいる。

50Kにおいては体積がより小さくなり、更に多くのペロブスカイト型構造を持つ物質とは異なる点としてG-type反強磁性が現れない。そして、メスバウアー効果の測定結果によると二種類の鉄原子が現れている。2:1で価電子3つと価電子5つを持つ鉄原子がそれぞれ出現し、電荷秩序状態に移行する。それらの鉄原子は強磁性的な$\rm{Fe}^{3+}-\rm{Fe}^{5+}$カップリングと反強磁性的な$\rm{Fe}^{3+}-\rm{Fe}^{3+}$カップリングを作っている。ただ物質全体としては反強磁性を示し、スピン密度波で記述される状態に移行している。

\section{批判}
議論の流れとして$\rm{Sr_{2}LaFe_{3}O_{8.94}}$の前に他のペロブスカイト構造を持つ物質をいくつか挙げて、それらの対比としてどのような物質かを論じている。しかし、他のペロブスカイト構造は所謂$\rm{AMO_{3}}$という典型的な構造を持っているのに対し、今回の物質はいささか複雑である。そこでそれまでされた議論をそのまま当てはめることができるのか、同じペロブスカイト構造であることを理由にそのまま対比していいのかどうかには疑問が出る。現に他のペロブスカイト構造に出現している現象が、今回の当物質には発言しないという例が見受けられる。
今回の論文では実験結果から結晶構造を導き出しているが、一方電荷秩序状態を明らかにするためにメスバウアー効果のデータを根拠に鉄原子の構造について述べている。しかし、本論文ではメスバウアー効果についてのデータが示されておらず、実のところは正しいのだが、それを根拠にした鉄原子の強磁性構造と反強磁性構造の証拠がないために真実かどうかが分からない。
結論としては室温と50Kでの実験データから結晶構造がどのように変化しているかを、X線回折と中性子散乱の二つの実験データから示している。このとき、各原子とその結合がどう変化しているかを定量的に示しているが、何故そうなるかが議論されていない。そしてそれが何次の相転移であるのかが考慮されていない。

\end{document}
