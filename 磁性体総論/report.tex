\documentclass{jsarticle}
\usepackage[dvipdfmx]{graphicx}
\begin{document}

\title{磁性体総論レポート1}
\author{1518511 川瀬 拓実}
\maketitle

\newpage

\section*{金属強磁性が生じる理由をstoner理論に基づいて出来るだけ数式を用いずに言葉で定性的に説明せよ。何故状態密度が大きいと強磁性になりやすいかも定性的に説明せよ。}
stoner理論によると、ハバード模型が適用できる系において磁場を印加すると、ゼーマン分裂が起きアップスピンとダウンスピンの状態密度にエネルギー差が生じる。このときフェルミエネルギー即ち電子の総数は変化しないのでアップスピン電子とダウンスピン電子の数に偏りができる。
するとアップスピンとダウンスピンの数が異なることで大局的な磁気モーメントが発生するので、この物質は以下のような磁化を持つ。
\begin{eqnarray}
  M = \mu_{B} \times \Delta N
\end{eqnarray}
ここで$\mu_{B}$はボーア磁子、$\Delta N$はアップスピンとダウンスピンの数の差を表す。
ハバード模型においてハーフフィルドの状態を仮定することで、具体的な電子数をフェルミエネルギーでの状態密度を使って以下のように書き表すことができる。
\begin{eqnarray}
  \Delta N = \frac{\mu_{B}H_{z}D(\varepsilon_{F})}{1 - \frac{U}{2N} D(\varepsilon_{F})}
\end{eqnarray}
ここで$H_z$は印加した磁場の大きさ、$D{\varepsilon_F}$はフェルミエネルギーの状態密度、$U$はクーロン反発エネルギーで電子相関の効果を表す項になる。
式(1)と(2)を合わせると以下のようになる。
\begin{eqnarray}
  M = \frac{\mu_{B}^{2} H_{z}D(\varepsilon_{F})}{1 - \frac{U}{2N} D(\varepsilon_{F})}
\end{eqnarray}
式(3)から、分母が0になるときに磁化が発散し物質が強磁性を示すことが分かる。

次に、発散するための条件を考える。式(3)の分母に着目すると、クーロン反発が大きい場合と状態密度が大きい場合がある。
クーロン反発エネルギーが極端に大きい場合を考えると、電子はもはや自由に動き回ることができなくなり原子の周りに局在することになる。すると原子1サイトあたりの電子数にもよるがMott絶縁体や電荷秩序絶縁体状態、いわゆる局在電子状態に転移してしまい強磁性状態は実現しなくなる。
すなわちフェルミエネルギー付近が禁制帯に含まれてしまい、多少磁場をかけたところで電子数に差が生じなくなってしまう。
以上のことから金属強磁性体になるための必要条件として状態密度が大きいことが挙げられる。


\end{document}
